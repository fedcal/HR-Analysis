\documentclass[a4paper]{extarticle}
\usepackage[utf8]{inputenc}
\usepackage{hyperref}
\usepackage{geometry}
\usepackage{fancyhdr}
\usepackage{graphicx} % libreria per le immagini
\usepackage{amssymb} %libreria per i simboli (ex. alfabeto reco)
\usepackage{algorithm2e} %libreria per scrivere pseudocodice
\usepackage{longtable}
\usepackage{caption}
\usepackage{lastpage}
\usepackage{tabularx}

\setlength{\parindent}{0em}%indentazione paragrafo
\setlength{\parskip}{1em}%spazio tra paragrafi
\renewcommand{\baselinestretch}{1.3}%interlinea
\graphicspath{ {./} }
\geometry{
    a4paper,
    left=10mm,
    right=10mm,
    bottom=20mm
}


\hypersetup{
    colorlinks=true,
    linkcolor=blue,
    filecolor=blue,      
    urlcolor=blue,
    pdftitle={Overleaf Example},
    pdfpagemode=FullScreen
}

\pagestyle{fancy}
\fancyhf{}
\rhead{Federico Calò}
\lhead{Ingegneria della Conoscenza - Relazione progettuale}
\cfoot{  \thepage }


\title{Ingegneria della Conoscenza - Relazione progettuale}
\author{\href{http://www.federicocalo.dev}{Federico Calò} }
\date{}

\begin{document}
\maketitle
\newpage
\tableofcontents
\voffset -30pt

\newpage

\section{Prefazione}

La stesura di questo documento è volta ad analizzare il progetto in Python sviluppato da \href{http://www.federicocalo.dev}{Federico Calò}, matricola 678191, per l'esame di Ingegneria della Conoscenza. Seguirà una breve introduzione del progetto, seguita dalle librerie utilizzate e dagli algoritmi scelti per la sua realizzazione.

\newpage

\section{Introduzione}

Questo progetto, dal nome di HR Analysis, nasce con l'intenzione di creare un programma per la gestione del personale di un'azienda informatica e 
dei relativi progetti. L'utente finale sarà una persona qualificata nella ricerca del personale, la quale dovrà interrogare 
la base dati per poter visualizzare i progetti attivi e il personale assunto. 

Tramite il sistema potrà quindi stabilire se:
\begin{itemize}
	\item Vi è un esubero di personale, ovvero se ci sono molte più persone rispetto ai progetti attivi
	\item Se vi è la necessità di assumere personale da impiegare in vari progetti
	\item Se vi è una situazione normale, ovvero in cui tutto il personale è impiegato nei vari progetti
\end{itemize}

Inoltre vi è la possibilità di interrogare il sistema in merito:
\begin{itemize}
	\item Alle competenze del singolo lavoratore
	\item Le relazioni tra i vari lavoratori
	\item Verificare se un determinato soggetto ha bisogno di formazione per migliorare alcune skill
	\item Verificare se un progetto può essere preso in carico in base alle persone presenti ed eventualmente avviare un percorso 
		  di assunzione
	\item Collocare in un progetto le persone appena assunte
\end{itemize}

Inoltre il sistema elabora le informazioni relative alle caratteristiche individuali e ai livelli di senority (anzianità) di ogni singolo dipendente già assunto e in base a queste determina automaticamente il livello di una persona assunta.

Per ogni soggetto vengono presi in considerazione le seguenti caratteristiche:
\begin{itemize}
	\item Hard skills, individuate come le competenze tecniche di cui necessita per lavorare
	\item Work Routine, individuate come tutte le competenze trasversali
	\item Innovation, ovvero la capacità di saper migliorare le proprie competenze e apprenderne delle nuove
	\item Coordination, la capacità di coordinare piccoli gruppi di persone
\end{itemize}

Queste caratteristiche vengono espresse con dei valori numerici da 1 a 5, dove ogni numero rappresenta una certa padronanza in quella determinata caratteristica.

Per le \textbf{hard skills} potremmo avere le seguenti valutazioni:

\begin{tabularx}{\textwidth} { 
  | >{\centering\arraybackslash}X 
  | >{\centering\arraybackslash}X 
  | >{\centering\arraybackslash}X | }
 \hline
 Valore & Livello & Descrizione \\
 \hline
 1  & Principiante  & Competenze di base relative al proprio lavoro.  \\
 \hline
 2  & Familiare  & Competenze adeguate per agire in modo autonomo  \\
 \hline
 3  & Competente  & Competenze avanzate in uno o più contesti produttivi  \\
 \hline
 4  & Esperto  & Competenze di alto livello  \\
 \hline
 5  & Master  & È il referente dell'argomento.  \\
\hline
\end{tabularx}

Per la \textbf{Work Routine} potremmo avere le seguenti valutazioni:

\begin{tabularx}{\textwidth} { 
  | >{\centering\arraybackslash}X 
  | >{\centering\arraybackslash}X 
  | >{\centering\arraybackslash}X | }
 \hline
 Valore & Livello & Descrizione \\
 \hline
 1  & Segue  & Eseguire attività di routine sulla base di istruzioni fornite da altri.  \\
 \hline
 2  & Applica  & Esegue compiti non di routine di media complessità  \\
 \hline
 3  & Capace  & Esegue compiti diversi, diversi in termini di complessità e in diversi contesti.  \\
 \hline
 4  & Sicuro  & Esegue attività complesse all'interno di ambienti imprevedibili  \\
 \hline
 5  & Influente  & Lavora in ambienti complessi e imprevedibili contribuendo alla definizione di linee guida per risolvere o
anticipare i problemi  \\
\hline
\end{tabularx}

Le valutazioni relative all'\textbf{Innovation} sono:


\begin{tabularx}{\textwidth} { 
  | >{\centering\arraybackslash}X 
  | >{\centering\arraybackslash}X 
  | >{\centering\arraybackslash}X | }
 \hline
 Valore & Livello & Descrizione \\
 \hline
 1  & Conprensione  & Osserva le innovazioni introdotte da altri, le comprende e ne può percepire i vantaggi o
svantaggi.  \\
 \hline
 2  & Applica  & Osserva e giudica l'innovazione introdotta da altri, applicandola efficacemente al proprio contesto operativo  \\
 \hline
 3  & Condivide  & Osserva e giudica l'innovazione introdotta da altri, applicandola efficacemente al proprio contesto operativo,
promuovendo l'adozione all'interno del team.  \\
 \hline
 4  & Promuove  & Promuove e suggerisce l'innovazione anche condividendo i risultati con i team e l'azienda  \\
 \hline
 5  & Ispira  & Ispira le persone a introdurre l'innovazione, influenza l'adozione e promuove le innovazioni a tutti i livelli.  \\
\hline
\end{tabularx}

Infine per quanto riguarda la \textbf{Coordination} sono:

\begin{tabularx}{\textwidth} { 
  | >{\centering\arraybackslash}X 
  | >{\centering\arraybackslash}X 
  | >{\centering\arraybackslash}X | }
 \hline
 Valore & Livello & Descrizione \\
 \hline
 1  & Da solo  & Riceve istruzioni, non ha persone da coordinare.  \\
 \hline
 2  & Team interno  & Si occupa di coordinare 1/2 stagisti da coinvolgere all'interno di un percorso di formazione sul lavoro.  \\
 \hline
 3  & Piccolo team  & Si concentra sul coordinamento di 1/3 junior che lo riconoscono come referente.  \\
 \hline
 4  & Team  & Si occupa di coordinare 3/5 persone con diversa anzianità, il coordinamento e la gestione
le abilità sono ben formate e di alto livello  \\
 \hline
 5  & Grande Team  & Sa gestire e coordinare persone diverse, diverse per anzianità, competenze, background.
Coordina più piccoli team (3/5) o un team più grande (6/12) \\
\hline
\end{tabularx}


Per navigare il sistema si pensa di utilizzare un'interfaccia grafica, mentre i dati verranno storicizzati in un database 
relazionale. I dati verranno caricati in memoria attraverso una struttura dati chiamata grafo. Ogni nodo del grafo sarà 
costituito da un dipendente e le relazioni tra i dipendenti saranno rappresentate dagli archi tra i nodi. Non ci saranno 
nodi isolati in quanto ogni dipendente avrà un responsabile e farà parte di un team di sviluppo o di una classe di formazione. 

Ogni progetto, per essere preso in carico, dovrà rispettare le seguenti caratteristiche:

\begin{itemize}
	\item Deve essere disponibile un Product Manager per la gestione complessiva del progetto
	\item Deve essere presente un analista funzionale, il cui compito è quello di formalizzare le richieste del cliente
	\item Deve essere presente un UX designer per la gestione delle interfacce grafiche
	\item Un gruppo di sviluppatori composto da
	\begin{itemize}
		\item 2 sviluppatori senior
		\item 2 sviluppatori middle
		\item 2 sviluppatori junior
	\end{itemize}
\end{itemize}


Un Product Manager può curare gli aspetti relativi di due progetti contemporaneamente, mentre tutti le altre figure possono 
partecipare allo sviluppo di un solo progetto.
Se non sono rispettate queste caratteristiche l'HR può riservarsi di inserire il progetto in uno stato di pending della 
durata di due settimane, al fine di poter assumere le figure professionali coinvolte.


\section{Librerire utilizzate}

\href{Vender}{https://github.com/cjhutto/vaderSentiment}


 
\end{document}



